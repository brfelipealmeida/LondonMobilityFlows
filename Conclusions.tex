\chapter{Conclusions}
\label{chapterlabel4}

In conclusion, this study has unveiled the intricate dynamics of urban mobility, offering insights into the evolving landscape of transportation modelling in contemporary cities. The primary focus of this research was to explore how Deep Gravity and Spatial Interaction Models perform in predicting mobility flows, with a specific emphasis on distinguishing between non-work-related and work-related trips.

Therefore, to address the research question, it was observed that the Deep Gravity Model excels in estimating mobility flows when dealing with a substantial volume of data. Conversely, the Spatial Interaction Model demonstrates strength when working with more aggregated data. Furthermore, the Deep Gravity Model consistently delivers impressive results when assessing the two models in the context of work and non-work flows, particularly compared to the Production-Constrained Model. However, the model reveals certain limitations for non-work flows, which may arise from either a simplified feature structure or the selected period, limited to a single day in this study.

A noteworthy aspect of this research is the innovative use of a private database containing the origin and destination data of individuals in London sourced from smartphone location information. This approach allows for an origin and destination analysis using exceptionally current data. In contrast, relying on Census data, last updated in 2011, would have provided a snapshot of a society undergoing rapid structural and social transformations over the past 12 years.

Today, the cities have evolved significantly with the widespread use of smartphones, advancements in deep learning techniques, and improved access to high-performance computing resources. Such advances enable individuals to conduct complex data analyses, even on powerful supercomputers, from the comfort of their own homes. This study was conducted on Google Colab, showcasing the accessibility of cutting-edge computational tools.

These developments converge to facilitate real-time analysis of city dynamics, a feat made possible through collaboration with private companies holding extensive data collections. This exploratory study represents an initial step in exploring these urban technologies, intending to develop increasingly sophisticated and effective models. Notably, the Deep Gravity model, a key focus of this research, was published in 2021. This study has focused on the predictive capabilities of Deep Gravity and Spatial Interaction Models, providing valuable insights into mobility flows and their differentiation between non-work and work-related trips.



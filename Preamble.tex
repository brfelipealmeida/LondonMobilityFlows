% I may change the way this is done in a future version, 
%  but given that some people needed it, if you need a different degree title 
%  (e.g. Master of Science, Master in Science, Master of Arts, etc)
%  uncomment the following 3 lines and set as appropriate (this *has* to be before \maketitle)
% \makeatletter
% \renewcommand {\@degree@string} {Master of Things}
% \makeatother

\title{Estimating Mobility Flows: Comparative Analysis of Deep Gravity Model and Spatial Interaction Model for non-work flows in London}
%\subtitle
\author{Felipe Santos Almeida}
%supervisors
\department{The Bartlett Centre for Advanced Spatial Analysis}

%\begin{center}
    % UCL IMAGE
%    \vspace*{-3.5cm}
 %   \hspace*{-1cm}
%    \makebox[\textwidth]{\includegraphics[width=\paperwidth]{Images/UCL_LOGO_new.png}}
%\end{center}
\maketitle
\makedeclaration


\begin{abstract} % 300 word limit

The dynamics of human mobility have experienced a profound transformation in recent years due to a confluence of factors. The interplay between transportation systems has shaped this evolution due to the increased urban population, the integration of computational techniques, advanced data collection methods, and Artificial Intelligence (AI). This dissertation examines the evolving nature of human mobility in London, focusing on work-related and non-work-related flows.

This study aims to compare and evaluate the performance of two models, namely the Deep Gravity Model and the Production-constrained Model, concerning their ability to predict mobility patterns in London. A comprehensive understanding of their efficacy and applicability is established by comparing the outcomes of these models against observed flow patterns. The study specifically concentrates on the London region, containing its diverse cultural, economic, and infrastructural facets. This comprehensive approach is crucial for comprehending the complex commuting patterns across the city.

Three key datasets are utilised to conduct this analysis: Mobility flows, Points of Interest, and Census 2021. Mobility data offers essential aggregated origin-destination flow data within the hexagons grid system, a unique hierarchical classification to each cell, serving as this study's foundational index. The additional datasets are integrated into this index through area-weighted spatial joins, ensuring data coherence.

The findings of this study are on the predictive capabilities of the Deep Gravity Model, highlighting its effectiveness in capturing and foreseeing human mobility patterns. The comparison between the Deep Gravity Model and the Production-constrained Model provides valuable insights into mobility dynamics, contributing to advancing transportation research and policy-making.

\end{abstract}

\begin{acknowledgements}

\end{acknowledgements}

\setcounter{tocdepth}{2} 
% Setting this higher means you get contents entries for
%  more minor section headers.

\tableofcontents
\listoffigures
\listoftables

